\sektion{3}{Implementation}

%%% OVERVIEW %%%
\subsection{Overview}

% System architecture figure
\begin{figure}[!htbp]
    \centering
    \includegraphics[width=0.59\textwidth]{Overview.png}
    \caption{High level architecture of \tigeruhr{}}
    \label{fig:architecture}
\end{figure}

The website was built using the Django Web Framework, and is currently hosted on Heroku. The backend is primarily written in Python, using Django, and the front end is written with standard web development languages: HTML, CSS, and Javascript. CAS is used for authentication, and PostgreSQL is used for database management. The app uses an Amazon S3 bucket for storing static files and media uploads. An interactive, RESTful API was also constructed using Django REST Framework, and documented using DRFDocs.

Behind the scenes, the app makes use of Heroku config variables, functionally equivalent to environment variables, to store sensitive information needed by the app, such as access keys and authorization parameters. For development, we also constructed a virtual environment with Vagrant that allows the user to run a developer build of the site on their machine, without installing any additional software.

%%% MODELS %%%
\subsection{Models and Database Design}

% TODO: Actually make this image
% Database design figure
%\begin{figure}[!htbp]
%    \centering
%    \includegraphics[width=0.59\textwidth]{Overview.png}
%    \caption{\textbf{THIS IS A STAND-IN PHOTO, REAL ONE TODO}}
%    % \caption{High level database design}
%    \label{fig:database}
%\end{figure}

Django models abstract database tables into python objects, which allows you to design and interact with your database in an object-oriented fashion, without worrying about writing specific queries \cite{django-models}. This is very convenient, especially considering we planned to build a RESTful API.

The database was designed with the API in mind: the two focal points of the app, the Resources and the Positions, were given their own database models, \texttt{Resource\_Profile} and \texttt{Position} respectively, and form the ``roots'' of the database. Most of the other models branch off them, and describe a more specific aspect about, or relationship between the more general root objects. The rest describe independent entities not necessarily tied to a resource or position.

%The core database models are illustrated in Figure \ref{fig:database}. A brief description of the most important ones are provided below:
A brief description of the most important database models are provided below:
\bullist
    \item \texttt{Resource\_Profile:} The root of the Resource model tree, this model contains the objective information for a student, including their associated \texttt{User}, netid, graduation year, picture, and transcript.
    \item \texttt{Resource\_Repute:} Contains information that is either inferred by our systems, or is uploaded by others, including \texttt{auto\_clear}, which indicates a student has been strongly recommended for hiring, academic standing, course remarks, and interview remarks.
    \item \texttt{Resource\_Position\_Info:} Contains information relating a resource to their time hired at that position. Includes the start date and end date, along with their performance rating and comments, if they exist. If the resource is currently hired, \texttt{end\_date} will be \texttt{None}
    \item \texttt{Resource\_Application:} Represents a resource's application to a position. Contains the time the application was submitted.
    \item \texttt{Position:} The root of the Position model tree, this model contains all critical information about a position, including its name, its slug (a unique identifier), the \texttt{User} corresponding to a staff contact, and tables of resources in various stages of the application process.
    \item \texttt{Position\_Offer:} Represents a hiring offer presented to a resource. Contains the time the offer was extended.
    \item \texttt{Remark:} Represents a general comment about a resource, and may be used for several topics. Stores the author and receiver of the comment, along with the time it was written.
    \item \texttt{Time\_Slot:} Represents a half-hour block of time used by the scheduler \seesec{Scheduler}. Stores the resource or position it is describing, and whether the time slot is ``active''. For Resources, ``active'' means the the student is free at that time, and for Positions, it means that block is required.
\finbullist

%%% AUTHENTICATION %%%
\subsection{Authentication and Permissions}

Users are required to login through CAS before being able to access the site. It is authenticated against Princeton's system, so users just use their standard netid and password. Authentication for the API is currently session based, and requires the user to log in using an account associated with a head manager or site administrator. Eventually, we plan to change this to use token based authentication instead.

Permission handling is performed using Django's default User permissions system. There are five permission levels, detailed below:
\bullist
    \item \texttt{Admin:} Has access to everything, and permission to perform everything.
    \item \texttt{Head Manager:} Has access to everything on the site, the API, and permission to add new Positions.
    \item \texttt{Manager:} Can view all information for resources, and can access management pages for the positions they control.
    \item \texttt{Assistant:} Can view abstracted information for resources, and can view, but not edit, management pages for the positions they assist.
    \item \texttt{Resource:} May only view and edit their own profile page.
\finbullist

%%% API %%%
\subsection{API and Documentation}

% TODO: Fix "Documentantion" typo in Picture. Also is this the best place for it?
% API documentation figure
\begin{figure}[!htbp]
    \centering
    \includegraphics[width=0.7\textwidth]{Documentation.png}
    \caption{Documentation of \tigeruhr{}'s API}
    \label{fig:documentation}
\end{figure}

A REST-compliant API was built using Django REST Framework \cite{django-rest-framework}. The framework works very well with databases designed in an object oriented manner: it provides a \texttt{Serializer} class that wraps each of your models, and allows their content to be parsed and rendered from several content types such as JSON. The serializers, and the provided class mixins, can then be used to autogenerate REST API methods, which can be customized by overriding them.

The framework also features a live endpoint system that allows to you to interact with the API like a typical website. When you visit an active API URL, such as \url{/api/v1/resources/} for example, you will be presented the \texttt{GET} data for that request presented in JSON format, along with parameters to send a \texttt{POST} or \texttt{PUT} request at the bottom as well.

Finally, we recognize the importance of having clear documentation for an API, so we used its companion library, DRF Docs \cite{drf-docs} to do that. Upon visiting \url{/docs}, you will be presented with all active API URLs, along with a description of what they do, which HTTP methods are allowed, and the parameters each one expects. An image of the documentation page is presented in Figure \ref{fig:documentation}.

%%% ALL RESOURCES %%%
\subsection{All Resources Table}

% All resources table figure
\begin{figure}[!htbp]
    \centering
    \includegraphics[width=0.9\textwidth]{Resources.png}
    \caption{All Resources Table}
    \label{fig:resources}
\end{figure}

Figure \ref{fig:resources} shows the ``All Resources'' table, viewable by hiring managers and assistants. Here, you can view, sort, and search through all registered resources at once. This table provides the identifying information for each student, their job status, and a general indication of their preferred time commitment and academic standing, without giving sensitive details away. The \textit{Academics} column on this page just shows a reflection of a student's average COS grades; eventually, we plan for hiring managers to be able to specify which courses are important, and how much weight each one holds for their own specific academic standing calculation.

Clicking on a resource's name takes you to their profile page, where you can see more detailed information \seesec{Profile Page}. Clicking on a table's column header allows you to sort by that column in both ascending and descending order.

\newpage
On the left side of the page, there is a navigation menu with links for managing specific courses. A hiring manager will only be able to see and access pages for the position they manage.

%%% MANAGEMENT %%%
\subsection{Management Tables}

% Example management table figure
\begin{figure}[!htbp]
    \centering
    \includegraphics[width=0.9\textwidth]{Applicants.png}
    \caption{Example Management Table}
    \label{fig:applicants}
\end{figure}

Figure \ref{fig:applicants} features an example management table for a position. Specifically this is the ``Applicants'' table, which you can reach by clicking on the nav link for a position, then clicking on ``Applicants'' in the menu that appears.

As explained in ``Approach'', hiring is a three stage process:
\numlist
    \item At the beginning of a hiring season, resources apply to all positions they're interested in.
    \item First round selection begins. Hiring managers can see the list of applicants for their position, and either extend them an offer immediately, reject them, or advance them to a second round, where additional information may be requested. This additional information may be in the form of an interview, or an online form for example.
    \item Resources deferred to the second round provide requested information, and the hiring managers review them. Then second round selection begins, where the hiring managers must choose between extending an offer or rejecting each student. Resources may either accept or reject position offers they receive.
\finnumlist

This is the table of first-stage applicants, so the three options of extending an offer, advancing them to the next round, or rejecting them are available. Using the table is very intuitive: just click on the resources you want to select, then press the appropriate button. The button press will asynchronously call the corresponding API function, and update the table.

In addition, there are three other management tables for each position:
\bullist
    \item \textbf{Hired:} Displays the resources currently holding the position.
    \item \textbf{Second Round:} Displays the resources who completed the second-round requirements.
    \item \textbf{Other:} Displays resources who have been extended an offer, but not replied yet, as well those who have been fired.
\finbullist

%%% REGISTRATION %%%
\subsection{Registration}

% Registration page figure
\begin{figure}[!htbp]
    \centering
    \includegraphics[width=0.9\textwidth]{Register.png}
    \caption{Resource Registration Page}
    \label{fig:registration}
\end{figure}

Figure \ref{fig:registration} displays the registration screen. Registration consists of two pages: the first, shown in the figure, asks the student to enter the number of hours they're willing to work in a week, which positions they would like to be considered for, and to upload their transcript. The next page has the resource fill out a WhenIsGood-esque schedule of their free time.

Thus, registration is incredibly quick, and the student has to provide very little information; everything else is automatically collected by the system. A number of submodules were needed to accomplish this:

%%% RESOURCE SCRAPER %%%
\subsubsection{Resource Scraper}

% Tigerbook figure
\begin{figure}[!htbp]
    \centering
    \includegraphics[width=0.8\textwidth]{Tigerbook.png}
    \caption{A Student's Page on Tigerbook}
    \label{fig:tigerbook}
\end{figure}

\newpage
First, the resource scraper, which acquires the student's full name, email address, class year, and prox photo from their netid. Most of a student's identifying information is publicly available, provided you have something to search for them by, and can authenticate into the site of your choosing. Figure \ref{fig:tigerbook} shows the information displayed about a student on Tigerbook, with a similar setup used for Princeton's College Facebook. The student's netid is retrieved when they authenticate through CAS, so all we need the module to do is actually perform the web scraping.

First, the script opens a connection to Princeton College Facebook, authenticates, and searches for the provided netid using python's \texttt{urllib2} library. It then uses \texttt{Beautiful Soup} to scrape the desired information from the results.

In addition, it creates a folder for the student in the app's Amazon S3 bucket, and stores their picture there. The module uses the \texttt{boto} library to establish a connection with the bucket, and the authentication parameters for both the bucket and Princeton College Facebook are stored inside Heroku configuration variables for security.

%%% TRANSCRIPT PARSER %%%
\subsubsection{Transcript Parser}

% Transcript parser state diagram figure
\begin{figure}[!htbp]
    \centering
    \includegraphics[width=0.8\textwidth]{TP-Flowchart.png}
    \caption{Simplified FSM of Official Princeton Transcript Parser}
    \label{fig:tp-flowchart}
\end{figure}

Next, the transcript parser, which extracts a student's grades from the transcript they upload. First the text is extracted, and the program verifies the transcript's authenticity. We wanted to support both official and unofficial Princeton transcripts, so the module then identifies which type of transcript it is, and calls the appropriate parser.

Both parsers use a state machine to keep track of the information it receives. Since there are certain invariants on the order in which information arrives, we can detect a parsing error if something arrives unexpectedly, or throw away data we know is unwanted. For example, Figure \ref{fig:tp-flowchart} shows a simplified FSM of the state machine used by the official transcript parser. For this type, we know the text will be extracted such that a course department will always come first, followed by the course number, then the grade received in the course, with junk data interspersed between the three pieces of information. We can throw away anything that isn't one of the three pieces, because it isn't important to us. If we receive one of the pieces out of order however, we know that something unexpected has occurred, and the parsing is no longer reliable, so we throw an exception and exit.

The module returns a dictionary relating each course to the grade received, and the semester it was taken.

%%% WHENISGOOD %%%
\subsubsection{Scheduler}

% Scheduler figure
\begin{figure}[!htbp]
    \centering
    \includegraphics[width=0.8\textwidth]{Schedule.png}
    \caption{Scheduler Page During Registration}
    \label{fig:scheduler}
\end{figure}

Finally, we designed a WhenIsGood-esque view for entering and displaying schedule information. We wanted users to easily tell what hours were required for a position, and input their available work hours as quickly and naturally as possible, so we modeled it after a WhenIsGood, since most people are familiar with how it works, and the design is very clean and easy to use. Our scheduler functions exactly the same way: you click and drag along the time slots your available, and it displays your selections in a slick, compact format. Figure \ref{fig:scheduler} shows the scheduler in action, on the second page of the registration form.

These modules were implemented as standalone programs that could be used outside of TigerUHR if desired.

%%% PROFILE PAGE %%%
\subsection{Profile Page}

% Profile page figure
\begin{figure}[!htbp]
    \centering
    \includegraphics[width=0.9\textwidth]{Profile-Manager.png}
    \caption{Profile Page while logged in as a Manager}
    \label{fig:profile-manager}
\end{figure}

Figure \ref{fig:profile-manager} shows the resource profile page, while logged in as a manager. Here, a manager can view the student's complete identifying information, schedule, grades, and comments on their previous interviews and job performances. Interviewers and assistants can also view resource profile pages, but sensitive information, such as the student's specific grades, will not be displayed. Managers and assistants will be able to add comments on the resource's job or interview performances, provided they have the proper privileges to do so.

Resources can also visit their own profile page. There, they can reupload their transcript, update their schedule information, or re-select which positions to be considered for. They can also accept offers and complete second round form requests. The resource will be notified via email whenever they have something to respond to, such as an offer or online form, so they don't have to keep checking the site.

