\documentclass[../report.tex]{subfiles}
\begin{document}
\onehalfspacing

\section{Conclusion}

Throughout this paper, we investigated the importance and feasibility of formally verifying a large scale system.
Google's Verified Boot program was shown to be a good candidate for formal verification because of its security guarantees and its interactions across the hardware firmware boundary.
Verified Boot's properties were expressed formally using CTL and many were verified against the C implementation using the model checker CMBC\@. 
Model checking abstractions were discussed, such as unimplemented functions, which helped make the model checking possible in a short time. 
Additionally, model checking was shown to be a good replacement for unit testing which is in place throughout many C systems.

\subsection{Future Work}

Throughout the next semester I hope to refine my security properties and gather more CBMC data now that I have the model checking framework running.
At the moment all hardware functions are abstracted as C programs, but this may provide different implementation mismatches between the simulation and a real platform.
I hope to implement previous work which is able to take hardware specifications and build an accurate implementation for model checking\cite{ila}.
In particular, a model of the TPM is necessary to fully verify that it can act as secure storage.
As TPMs are widely used in systems for various security protocols, an ILA model may be useful beyond the scope of this project.


\end{document}
