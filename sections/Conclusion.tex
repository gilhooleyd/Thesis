\chapter{Conclusion}

This paper investigated the feasibility of formally verifying a large scale system.
Google's Verified Boot program was shown to be a good candidate for formal verification because of its security guarantees and its interactions across the Hardware-Software boundary.
VBoot's security claims were expressed formally using CTL and were verified against the C implementation using the model checker CMBC\@. 
Model checking abstractions were discussed, such as abstracted functions, memory
allocation, and non-deterministic array writes.
These abstractions were necessary for Formal Verification tools to produce
results on the Verified Boot software.
It was shown that the high RAM usage of CBMC prevents verification from being run on personal computers.

\section{Future Work}

The analysis on Verified Boot is helpful in the way that it can be applied to
other systems. 
Verified Boot was chosen because it is a widely used code base
that crosses the Hardware-Software boundary. 
However other systems exhibit more interesting behavior like concurrent hardware accesses.

Verified Boot uses hardware, but there are very little opportunities for
problems involving hardware concurrency.
Other systems include interweaving uses of non-blocking
Hardware\cite{load-protocol} and applying the ILA toolchain to these systems would showcase more strengths of the ILA.

Another direction for the ILA toolchain that has been considered but not explored
is its use to verify that two hardware modules have the same functionality.
This type of verification  is important as a TPM can be produced by various Hardware
Vendors\cite{atmel-tpm}\cite{broadcom-tpm}\cite{infineon-tpm}, and the different 
implementations have been found to vary significantly\cite{tcg-inside}.
A full ILA description of a TPM would be very useful for comparing two hardware
chips from different vendors, as well as verifying that a software
implementation matches the Hardware implementation. 
Once the entire TPM functionality exists in an ILA description, more
verification could be done on the TPM API
itself\cite{TPM-state}\cite{TPM-spec-verif}.

Additionally, there are more areas in Google's Verified Boot that
deserve more attention. 
The Read-Write section of Verified Boot is ignored because it shares the
majority of its code with the Read-Only section, but it is still worth verifying. 
The Update functionality of Verified Boot is also be worth investigating.
Problems in the Update functionality will not affect the properties outlined in this report.
This report already assumes that an attacker can upload arbitrary updates.

Each of these suggestions are further areas of research that can be taken into Verified Boot or the field of Formal Verification.
