\documentclass[../report.tex]{subfiles}
\begin{document}
\onehalfspacing

\section{Conclusion}

Throughout this paper, we investigated the importance and feasibility of formally verifying a large scale system.
Google's Verified Boot program was shown to be a good candidate for formal verification because of its security guarantees and its interactions across the hardware firmware boundary.
Verified Boot's properties were expressed formally using CTL and many were verified against the C implementation using the model checker CMBC\@. 
Model checking abstractions were discussed, such as abstracted functions, memory
allocation, and non-deterministic array writes.
These abstractions were necessary for Formal Verification tools to produce
results on the Verified Boot software.
Additionally, model checking was shown to be a good replacement for the unit
testing assertions which is already in place throughout Verified Boot.

\subsection{Future Work}

The analysis on Verified Boot is helpful in the way that it can be applied to
other systems. 
Verified Boot was chosen because it is a widely used code base
that crosses the Hardware Firmware boundary. 
Now that it has been shown that analysis can work on real systems, academic
work can be done on systems with more interesting behavior.
Verified Boot uses Hardware, but there are very little opportunities for
problems involving Hardware concurrency.
Other systems include interweaving uses of non-blocking
Hardware\cite{load-protocol} and these systems would showcase more strengths of the ILA functionality. 

One direction for the ILA toolchain that has been considered but not explored
is its use to verify that two hardware modules have the same functionality.
This direction is important as there are many TPM's produced by various Hardware
Vendors\cite{atmel-tpm}\cite{broadcom-tpm}\cite{infineon-tpm}, and the different 
implementations have been found to vary significantly\cite{tcg-inside}.
A full ILA description of a TPM would be very useful for comparing two hardware
chips from different vendors, as well as verifying that a software
implementation matches the Hardware implementation. 
Once the entire TPM functionality exists in an ILA description, more
verification could be done on the TPM API
itself\cite{TPM-state}\cite{TPM-spec-verif}.

Additionally, there are more areas in Google's Verified Boot that
deserve more attention. 
The Read-Write section of Verified Boot is ignored because it shares the
majority of its code with the Read-Only section, but it also contains a few
possibilities for attacks.
The Update functionality of Verified Boot should also be worth mentioning,
although this report already assumes that an attacker can upload arbitrary
updates.

In the future it is possible to extend the techniques in this paper to the rest of
Verified Boot or other exciting academic projects.




\end{document}
