A security system would benefit greatly if there existed a proof that the system is bug free. 
This is done with the use of Formal Verification tools, which can analyze systems and provide formal proofs of correctness.
However, security systems are becoming larger and more complicated, making it harder to provide such proofs. 
These systems have also started to offload calculations to hardware modules.
In the past, tools for Formal Verification were focused on proving correctness for either a hardware system or a software systems.
Traditional verification tools are not equipt to handle current systems that depend heavily on interactions between hardware and software.
New tools have been developed to perform verification on these systems, but the tools are tested on small, academic systems.
This report investigates the use of new Formal Verification tools on the industry-sized security program Verified Boot.
Google's Verified Boot is a secure bootloader used in the Chromebook laptops.
Verified Boot uses cryptographic functions to ensure that only Google's code is run on the laptop.
This process involves many hardware-software interactions, making it a difficult system for traditional verification techniques.
In order to perform the verification, a new hardware abstraction tool is used to model Verified Boot's hardware.
Formal Models were created for sections of a Trusted Platform Module and for a full SHA accelerator.
Additionally, multiple software abstraction techniques are used in order to produce proofs on Verified Boot's large codebase in a reasonable time.
