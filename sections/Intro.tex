\chapter{Introduction}

Computing systems are becoming larger and more complicated, making security
vulnerabilities harder to find.
At the same time security breaches are having a larger impact on both personal and societal levels.
As the computer industry begins to produce laptops, smart phones, servers, and embedded systems, computer
hardware becomes more diverse and specialized. Diverse hardware increases the
potential for vulnerabilities, and a hardware vulnerability is difficult or
impossible to fix without replacing the device. 

The current industry standard is to use Unit Testing to discover errors in programs. 
Unit Testing is the process of dividing a program up into functional chunks and
verifying each function as a single unit. 
These units are used as ``black boxes'', meaning that they are fed a range
of test inputs, and the behavior is compared against expected results.
However, Unit Testing has a problem with scaling and with interface interaction.
Inputs for tests scale exponentially, meaning computers can only check a unit against
every possible input if the space of inputs is relatively small. 
Most Unit Tests only test against a few inputs that are thought to be
representative. 
However, this means that there is always the possibility of bugs
on the set of inputs that were not used in testing.
It also leads to the possibilities of false positives when testing on inputs that cannot appear in a real system.

Formal Verification can be seen as an enhancement to Unit Testing. 
Formal Verification is the act of proving or disproving the correctness of a
function by using mathematical proof techniques. 
This verification requires that the function's correctness is specified in a
series of formal properties.
Once these properties are created, current verification methods 
will produce proofs that the properties hold under all possible inputs.
Unlike Unit Testing, Formal Verification tools look at the code itself to quickly find inputs that would violate a user-defined property. 

Formal Verification guarantees that a program is working correctly. 
However, today's security programs are relying both on software programs and
hardware modules working together, and there are no industry standards for
 verification across this boundary.

\section{The Hardware-Software Boundary}

Originally a computer was a general purpose CPU, and software would
describe how the CPU should perform various tasks.
Today, computer engineers realize that specialized hardware can perform a single
task much more efficiently than generalized hardware.
This lead to the creation of a System on a Chip, or SoC. An SoC is the
combination of CPUs with smaller hardware modules that perform specialized
tasks.
The CPU is still the main processing unit, meaning it is responsible for
communicating with each hardware module.
However, the CPU no longer performs all processing in the system.

Hardware modules are taking some responsibility away from software.
If software wants to decrypt a message, then it can pass the message to an RSA
hardware module and receive the decrypted message back.
This means that the software does not need to implement a decryption function.
However, the software now needs to know how to communicate with the Hardware
module. 
Software that is aware of specific Hardware is known as Firmware.
Firmware is Software that reads and writes to registers connected to Hardware located on the SoC.
The remainder of this report will not use the term Firmware, and will instead refer to the boundary of Software and Hardware.
This is a decision to use a clearer naming convention.

The boundary between Software and Hardware is more important now that SoCs are becoming larger and including more
specialized Hardware. 
More computers are including security based hardware, like hashing and encryption/decryption modules.
It is no longer enough to verify software and hardware separately, as the
interactions between the two are an important part in modern security
algorithms.

\section{Existing Verification Techniques}

Formal Verification is defined as the application of mathematical methods to
prove or disprove the specification of a system~\cite{greenstreet}.
To use Formal Verification, properties of a system have to be specified by the
user. 
Propositional logic is one of the main ways that properties are specified.
For example, a programmer could express that given the fact that a password
check failed, the user should not be allowed to access bank statements.
If $A$ is access and $P$ is correct password, then the following statement
describes the property that, ``If access is given, then the Password is correct''.

\begin{equation}
    A \to P
\end{equation}

Propositional Logic statements can be described completely
through Boolean Logic and can then be evaluated using Boolean Satisfiability
Solvers (SAT solvers)~\cite{validating-sat}. 
SAT Solvers are an important component of verification but they have known limitations with scaling.
Additional systems have been used to extend the capabilities of SAT solvers, allowing them to describe more complicated properties and systems. 
One system is known as Satisfiability Modulo Theories (SMT), where 
Boolean variables can represent predicates such as inequalities between real valued variables (eg:
$x < y$)~\cite{smt}. 

SMT and SAT Solvers are two tools that perform Formal Verification. 
Automated tools exist to transform software programs with assertions into SMT statements.
This means that software can be easily verified with SMT solvers.
One popular tool that was used in this research is the C Bounded Model Checker (CBMC)~\cite{cbmc}. 
This tool automatically transforms software into SMT statements and verifies
user assertions.
More information about Verification tools can be found in
Section~\ref{sec:Verif}. 

\section{Problem Statement}

The trend in modern Computer Architecture is to offload computation to specialized hardware for either speed or power savings~\cite{hardware-accel}.
As this pattern increases the job of verifying functionality becomes more difficult.
Software and Hardware can be verified as separate units but this misses bugs
that exist as a result of the interaction between the two.

In past research, small SoCs have been created for the purpose of testing formal verification techniques~\cite{elane}.
These SoCs have been built by connecting numerous open source components where the hardware specifications are readily available.
Formal verification has been seen to be successful in analyzing security
protocols such as Secure Boot~\cite{elane}, Remote
Attestation~\cite{trustfound}, and Firmware Loading~\cite{load-protocol}. 
However, most research papers create their own SoCs from Open Source hardware
and develop their own firmware for verification.
While these papers are useful, they do not say much about the application of
Formal Verification to industry SoCs or industry Firmware.
SoCs and Firmware used by industry are typically much larger and more
complicated, which makes them more challenging for Formal Verification.
Additionally, other Formal Verification papers design techniques specific to a
single application, without considering the extension of a framework for
general Hardware Software verification.

A standard tool for verification across the Hardware Software boundary would
greatly speed up the verification of different systems.
Throughout this paper, the  Instruction Level Abstraction (ILA) toolchain will be used to define Hardware Software interactions.
An ILA defines Hardware as a collection of instructions that can be used by
Software through accessing specific registers~\cite{ila}.
The ILA toolchain provides a way to ``precisely capture updates to all firmware-
accessible states spanning cores, accelerators, and peripherals~\cite{ila-template}.''
This technique claims to allow software and hardware to be verified simultaneously and at scale using formal methods.

This thesis will be applying formal verification techniques to Google Chrome's
Verified Boot.
Verified Boot is a large scale, security based program that spans the Hardware
Software boundary. 
It is an implementation of a secure bootloader that has been designed by Google to run on their Chromebook laptops~\cite{vboot}.
A bootloader is responsible for initializing the computer's hardware and
loading the rest of the Operating System.
A secure bootloader will only load an OS that passes specific
security checks~\cite{secure-bootloader}.
Verified Boot checks that the OS it is loading has been signed by Google and is
unmodified.

Verified Boot is a good candidate for this report because its execution relies
on both Hardware and Software.
The Verified Boot code is also hard-wired into the Chromebook laptops, meaning
that any bugs must be found before laptop production starts.
The importance of finding bugs makes Formal Verification guarantees more appealing.
Additionally, Google has released both Verified Boot and Chrome OS as Open
Source software, meaning verification can be run directly on industry code.

\subsection{Contributions}

The main contribution of this thesis is the application of Formal Verification
tools to a large industrial project that spans the Hardware Software boundary.
In order to perform Verification, specific abstractions are used, and these
techniques can be extended to other large projects.

This report also takes advantage of the Instruction Level Abstraction tool chain
to show its effectiveness in a large scale system.
In doing so, a hardware model is created for the Trusted Computing
Module (TPM).
The universal nature of the ILA toolchain means this hardware abstraction can be used in other projects.
The use of the ILA toochain in this paper also displays its effectiveness in large projects.

Finally, this report ends with security recommendations for Verified Boot, which
would harden the boot system based on the results of the Formal Verification.

\section{Overview}

This thesis moves from a description of the Verified Boot system to the Formal
Verification techniques.
Chapter 2 describes the Hardware used (and not used) by Verified Boot.
Chapter 3 describes the Software implementation and organization.
It describes any software deviations from Verified Boot that were
necessary.
Chapter 4 describes the Formal Verification techniques and abstractions used.
Chapter 5 outlines the results of Formal Verification and outlines security
recommendations for Verified Boot.
Chapter 6 concludes the thesis by reiterating results and exploring
possibilities for future work.
